\documentclass[10pt,onecolumn,letterpaper]{article}
\usepackage[top=1in, bottom=1.25in, left=1.25in, right=1.25in]{geometry}

\setcounter{page}{1}
\begin{document}

%%%%%%%%% TITLE %%%%%%%%%%%%%%%%%%%%%%%%%%%%%%%%%%%%%%%%%%%%%%%%%%%%%%%%%%%%%%%%
\title{Combinatorial Auctions}

\author{Nathan Immerman\\
College of Engineering, University of Michigan\\
Ann Arbor, Michigan\\
{\tt\small immerman@umich.edu}
}

\maketitle

%%%%%%%%%% Body %%%%%%%%%%%%%%%%%%%%%%%%%%%%%%%%%%%%%%%%%%%%%%%%%%%%%%%%%%%%%%%%

\section{Abstract}

\section{Introduction}

In a standard auction, there is a single good being sold to one or more players who have a demand for that good. The auctioneer determines which player is allocated the good based on which player displays the greatest demand. There are many ways for the auctioneer to coordinate the auction, including the ascending and descending price auction, which all have the overall goal of determining who is allocated the good. Combinatorial auctions generalize this concept by allowing multiple goods to be auctions and players have demands for bundles or subsets of the goods. 

Formally, each player $i$ in the combinatorial auction has a valuation function $v_i$ which describes their value for all subsets of goods in the auction. For each subset of goods $S$, player $i$ receives the value $v_i(S)$ if they receive the bundle. A player's valuation must be monotone, namely if $S \subseteq T$ then $v_i(S) \leq v_i(T)$ and must be normalized to $v_i(\emptyset)= 0$. By defining a player's valuation function in this manner we allow players to express complementary and substitutive goods in their valuations. Completment goods are valued more together that the sum of their values individually, $v(S \cup T) > v(S) + v(t)$, while substite good are valued more individually, $v(S \cup T) < v(S) + v(t)$. By allowing such expressions, players can fully expresses their demands for many sets of goods.

In combinatorial auctions, the auctioneer has to determine allocation of goods for which any good is only allocated to one player, so every allocaiton is in the form $S_1,...,S_n$ where $S_i \cap S_j = \emptyset$ for every $i \neq j$. From this, the social welfare of the allocation is equal to $\sum_i v_i(S_i)$.

\section{Applications of Combinatorial Auctions}

Auctions are used to sell goods when the true values that the players have for the goods are unknown by the seller. Combinational auctions add the additional complexity of players values are based on a set of goods rather than just on a single good. This additional layer of complexity makes combinatorial auctions well suited for many applications including: the london bus routes, airport runway slots, and the Federal Communications Commision (FCC) airwave spectrum auctions. 

When the London bus system was deregulated in 1984, a problem was introduced as to how best to commision different companies to operate the different bus routs. Ultimately the commision in charge decided to use a form of a combinatorial first price aution to commision the bus routes. They made this decision so companies could take advantage of economies of scale and bid on multiple bus routes at a time. There are many benifits to owning similar bus routes, including lower fixed costs, so by using a combinatorial auciton the commision allowed companies to bid on bundles of routes without requiring the company to also bid on the individual routes. By using a combinatorial auction, the commision made it possible to create a more socially optimal allocation of the bus routes. 

Another application is the auctions of aiport runway slots. Throughout the course of a day, at a sinlge airport there is a limited number of runway time slots, when an airplane can either takeoff or land. The number of slots is based on the size of the airport, weather conditions, the sizes of airplanes and many other factors. The auctioning of these slots are well suited for a combinatorial auction because airlines have to posses many other goods to be able to utilize the runway slot. For an airline to use the runway slot they also have to have purchaes the airplane gate at the terminal, baggage servies, a landing runway slot at the destination airport and many other considerations. These interdependencies make runway slots idealy suited to be aucitoned in a combinatorial auction setting. 

Lastly, the FCC is moving towards using a combinatorial auction to sell the rights to used different spectrum bands in different geographical areas. Spectrum auctions are well suited for combinatorial auctions because for most companies having similiar bands in different areas is benificail. By using the same band in many locations, companies are able to create their hardware components more speciallized for that band. Ideally, by moving towards a combinatorial auciton, the FCC can create more optimal allocations that aids in the development of the use of wireless communication.




\section{Bibliography}



\end{document}
